\documentclass{physics_article}
\usepackage[T1]{fontenc}
\usepackage{comment}
\usepackage{physics}
\usepackage{bm}
\usepackage{cite}
\mytitle{Quantum Many-Body Ground States via Quantum Circuits }
\myname{Luca Petru Ion}
\studentid{20177336}
\date{}

\begin{document}
	\maketitle

	\section{Introduction \label{intro}}
	Quantum mechanics has not only revolutionized science and how we think of nature, but more recent in the second quantum revolution \cite{https://doi.org/10.48550/arxiv.quant-ph/0206091}, it has opened the scene for quantum information and quantum computing. At the moment we find ourselves in the NISQ (Noisy Intermediate-Scale Quantum) era \cite{Preskill2018quantumcomputingin} which means that we have quantum technology made up of 50 to few hundredths of qubits (intermediate) which, at the moment, cant be controlled precisely (noisy).NISQ presents us with many new technologies and practices like quantum cryptography \cite{bernstein_buchmann_dahmen_2009} and quantum deep learning \cite{https://doi.org/10.48550/arxiv.1711.02038}. In order to run, test and develop new quantum technologies and algorithms one must first have a realisation of a quantum computer. Many proposals, using different physical systems, have been provided: qubits via superconducting materials \cite{PhysRevLett.85.2208, nakamura_pashkin_tsai_1999}, quantum computation using Rydberg atoms \cite{PhysRevLett.74.4091,PhysRevLett.75.4714}, photonic quantum computing \cite{knill_laflamme_milburn_2001,doi:10.1126/science.abe8770} and silicon based quantum computers \cite{kane_1998,madzik_asaad_youssry_joecker_rudinger_nielsen_young_proctor_baczewski_laucht_et}. Quantum computation can also be used to simulate physical systems \cite{feynman_1982}. One area of physics that greatly benefits from this is the study of multi-body systems and as a result of this, the study of condensed matter physics. In our project we will tackle and study how quantum circuits can be used to find and approximate ground states of such systems.

	The complexity of quantum many-body physics lies in the fact that the dimension of the underlying Hilbert space grows exponentially as $p^N$, where here $p$ is the degrees of freedom of each of the individual particles and $N$ is the number of particles in the system. For most purposes we will consider two level systems $p = 2$, then we get a $2^N$ dimensional Hilbert space where a state is the superposition of the $2^N$ basis states which themselves are tensor products of the individual basis states of the two level sub-systems.The coefficients in the fourier decomposition of the state can be regarded as a tensor of rank $N$ where each index of the tensor is either $0$ or $1$ corresponding to the two degrees of freedom of the individual particles.This then opens the field of Tensor Networks (TN) \cite{orus_2014} which firstly gives us a tool to easily visualise quantum many-body systems in terms of TN diagrams and secondly, it opens the scene for programming code to optimize TN such that quantum many body problems can be solved efficiently numerically. As a side remark, MPS (matrix product states) is just a special case of TN for one dimensional systems; MPS, as explained in the next paragraph, also has had much success as a tool for certain algorithms.

	Currently the two prominent classical methods in use for quantum many body physics problems are the MPS and DMRG (density-matrix renormalization group). DMRG is one of the most powerful algorithms for studying ground states of one dimensional quantum lattices \cite{schollwock_2005,hallberg_2006}. It turns out that DMRG can be linked to MPS formalism \cite{https://doi.org/10.48550/arxiv.quant-ph/0608197} in fact DMRG can be formulated in the language of MPS as studied extensively in this article \cite{schollwock_2011}. The success of DMRG comes from the fact that ground states are not very entangled and therefore can efficiently be written in terms of MPS \cite{10.21468/SciPostPhysLectNotes.5}. DMRG was also extended to simulate real time evolution, however the entanglement of pure bipartite states increases linearly with time leading to a fast exponential growth in computational costs \cite{10.21468/SciPostPhysLectNotes.5}.More generally, the framework of TN is extensively used in classical simulation of quantum many-body systems where DMRG is just an example of a TN method for 1d quantum lattices. Multiple different TN methods exist designed to solve problems in different dimensions, an example is the PEPS (Projected Entangled Pair States)\cite{orus_2014}. 

	In an attempt to overcome some of the shortcomings of the classical methods, quantum methods for solving optimization problems and multi-body quantum systems were introduced. The Variational quantum eigensolver (VQE) was used by IBM to find the ground state energy of molecules in  quantum chemistry \cite{kandala_mezzacapo_temme_takita_brink_chow_gambetta_2017} and it was also implemented on a photonic quantum processing unit \cite{peruzzo_mcclean_shadbolt_yung_zhou_love_aspuru-guzik_obrien_2014} to solve problems involving small molecules. Another algortihm based on quantum computation is the QAOA (quantum approximate optimization algorithm) which was used to produce approximate solutions to combinatorial optimization problems \cite{https://doi.org/10.48550/arxiv.1411.4028} and found to outperform other methods \cite{PhysRevX.10.021067}. Recently, quantum circuit Tensor Networks were proposed, where TN like MPS and MERA (multi scale entanglement renormalization Ansatz)\cite{vidal_2008} were promoted to their quantum circuit versions, it is found that the quantum circuit TN are more expressive than their classical counterparts for certain physical models \cite{haghshenas_gray_potter_chan_2022}. The utility of quantum circuit TN has also been demonstrated by being implemented on NISQ machines for quantum simulations achieved through parallelism \cite{barratt_dborin_bal_stojevic_pollmann_green_2021}. The translation of TN to the quantum version poses the question if all progress in classical simulations of quantum systems can be translated in this way and would provide advantages over the classical methods. Most of the quantum methods described above give performance speed-ups compared to competing classical methods solving the same problem; in the case of multi-body quantum systems studies this is parlty due to the fact that quantum circuits that run on quantum computers, deal with a lot of entanglement fairly easily which is profitable for solving such systems numerically, whereas classically, a high amount of entanglement makes the classical methods struggle. On the contrary, not much is yet known on the limitations of the quantum algorithms via quantum circuits. Since the quantum algorithms solve the same problems as their classical counterparts and both these depend on the same number of parameters we have to assume that the quantum algorithms have certain limitations as well. In the case of multi-body systems, entanglement is one of the limitations for the classical methods, however in the quantum methods entanglement is efficiently generated, thus it means some other properties must carry the limitations. We are interested in finding these limitations and quantifying them by finding relevant bounds. 

	We seek to analyse these systems by first solving the eigenvalue problem for certain Hamiltonians and looking at the exact ground states. To do this we will consider multiple models: ising (free fermion), XXZ (integrable) and ising with transverse and longitudinal field (non-integrables). These were used as benchmarks in the literature \cite{PhysRevResearch.4.L022020,PRXQuantum.2.010342,PhysRevResearch.4.033118} thus form convenient examples. To solve these, we will adopt brute force exact numerical methods for small systems. We then model certain quantum circuits using python and use these to solve an optimization problem, namely we want to optimize the parameters in the quantum circuits to maximize the overlap of the resulting states with the exact states (fidelity). Optimization of fidelity is carried out to find the most optimal quantum circuit described by unitaries; an example of this optimization procedure is described in appendix B of \cite{PRXQuantum.2.010342}. We will use the exact states found out using the methods above, however other target states can be used which can be constructed through other classical methods like MPS, and those can be used instead via certain algorithms \cite{PRXQuantum.2.010342}. We will then numerically calculate different correlation functions and other objective functions for the exact states and for the quantum circuit approximations, and study how the results differ from each other. For example we might look at the entanglement entropy \cite{PhysRevResearch.4.033118}. We will plot the results to show the deviation of the ground states generated by our quantum circuit ansatz, to the exact states; plots of this form for the ising model using a sequential ansatz were already provided in \cite{PRXQuantum.2.010342}. Moreover, We will look at different circuit architecture and do all the above for these different circuit structures and record the differences in the results. We would like to ...


 	\bibliography{refs}
 	\bibliographystyle{unsrt}
\end{document}