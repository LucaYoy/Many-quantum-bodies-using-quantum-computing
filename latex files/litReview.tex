\documentclass{physics_article}
\usepackage[T1]{fontenc}
\usepackage{comment}
\usepackage{physics}
\usepackage{bm}
\usepackage{cite}
\mytitle{Literature Review on Many-Body Ground States via Quantum Circuits }
\myname{Luca Petru Ion}
\studentid{20177336}
\date{}

\begin{document}
	\maketitle
	\tableofcontents

	\section{Introduction \label{intro}}
	Quantum mechanics has not only revolutionized science and how we think of nature, but more recently ,in the second quantum revolution \cite{https://doi.org/10.48550/arxiv.quant-ph/0206091}, it has opened the scene for quantum information and quantum computing. The NISQ era \cite{Preskill2018quantumcomputingin} presents us with many new technologies and practices like quantum cryptography \cite{bernstein_buchmann_dahmen_2009} and quantum deep learning \cite{https://doi.org/10.48550/arxiv.1711.02038}. In order to run,test and develop new quantum technologies and algorithms one must first have a realisation of a quantum computer. Many proposals, using different physical systems, have been provided: qubits via superconducting materials \cite{PhysRevLett.85.2208, nakamura_pashkin_tsai_1999}, quantum computation using Rydberg atoms \cite{PhysRevLett.74.4091,PhysRevLett.75.4714}, photonic quantum computing \cite{knill_laflamme_milburn_2001,doi:10.1126/science.abe8770} and silicon based quantum computers \cite{kane_1998,madzik_asaad_youssry_joecker_rudinger_nielsen_young_proctor_baczewski_laucht_et}.

	With all these emerging technologies, optimization problems and multi-body quantum systems are a topic of great interest. The Variational quantum eigensolver (VQE) was used by IBM to find the ground state energy of molecules in  quantum chemistry \cite{kandala_mezzacapo_temme_takita_brink_chow_gambetta_2017} and it was also implemented on a photonic quantum processing unit \cite{peruzzo_mcclean_shadbolt_yung_zhou_love_aspuru-guzik_obrien_2014} to solve problems involving large systems of molecules. Other algorithms suited for different purposes are: the QAOA (quantum approximate optimization algorithm) which was used to produce approximate solutions to combinatorial optimization problems \cite{https://doi.org/10.48550/arxiv.1411.4028} and found to outperform other methods \cite{PhysRevX.10.021067}, the DMRG (density-matrix renormalization group) which is one of the most powerful algorithms for studying one dimensional quantum lattices \cite{schollwock_2005,hallberg_2006}. It turns out that DMRG can be linked to MPS (matrix product states) formalism \cite{https://doi.org/10.48550/arxiv.quant-ph/0608197}, a powerful way of rewriting a quantum state in a multi multi body system, in fact DMRG can be formulated in the language of MPS as studied extensively in this article \cite{schollwock_2011}. In the recent years a lot of research was made concerning the above, and many other quantum algorithms. Most of these methods give performance speed-ups compared to competing classical methods solving the same problem; in the case of multi-body quantum systems studies this is due to the fact that quantum circuits that run on quantum computers, deal with a lot of entanglement fairly easily which is profitable for solving such systems numerically, whereas ,classically, a high amount of entanglement makes the classical methods struggle. On the contrary, not much is yet known on the limitations of the quantum algorithms via quantum circuits. Since the quantum algorithms solve the same problems as their classical counterparts and both these depend on the same number of parameters we have to assume that the quantum algorithms have certain limitations as well. In the case of multi-body systems, entanglement is the limitation for the classical methods, however entanglement is effectively unlimited in the quantum methods, thus it means some other property must carry the limitations.  

	As hinted in the above paragraphs, our interest lies in solving multi-body quantum systems. We seek to analyse these systems by first solving the eigenvalue problem for certain Hamiltonians and looking at the exact ground states. To do this we will consider multiple models: ising, XXZ and ising with transverse and longitudinal field. These were used as benchmarks in the literature \cite{PhysRevResearch.4.L022020,PRXQuantum.2.010342,PhysRevResearch.4.033118} thus form conventional examples. To solve these, we will adopt brute force and numerical methods. We then model certain quantum circuits using python and use these to solve an optimization problem, namely we want to optimize the parameters in the quantum circuits to maximize the overlap of the reuslting states with the exact states (fidelity). Optimization of fidelity is carried out to find the most optimal quantum circuit, described by unitaties. We will use the exact states found out using the methods above, however exact states can be found through other methods like MPS, and those can be used instead via certain algotithms \cite{PRXQuantum.2.010342}. //We will then numerically calculate different correlation functions and other objective functions for the exact states and for the quantum circuit approximations, and study how the results differ from each other. We will look at different types of circuits and do all the above for these different circuit structures and see how our results differ.

	\begin{comment}
		In this article We will start with an overview{} of the relevant quantum mechanical techniques and mathematics which forms the basis of the project, we will then follow up with a description of the computational software and techniques that will be used, and finally we will discuss some of the work already done in the field.

		\section{The mathematics of quantum information}
		\subsection{The Hilbert space}
		Any quantum system is described by a state $\ket{\psi} \in \mathcal{H}$ in a Hilbert space $\mathcal{H}$. In quantum information we are usually interested in $\mathbb{C}^2$ with basis vectors ${\ket{0},\ket{1}}$, which are the eigenstates of $\sigma_z$ Pauli matrix. For convenience below are the three Pauli matrices written in the above basis.
		\begin{align}
		\sigma_x &= \begin{pmatrix}
		 	0&1\\
		 	1&0
		 \end{pmatrix}\\
		 \sigma_y &= \begin{pmatrix}
		 	0&-i\\
		 	i&0
		 \end{pmatrix}\\
		 \sigma_z &= \begin{pmatrix}
		 	1&0\\
		 	0&-1
		 \end{pmatrix}
	 	\end{align}
	 	we shall not discuss here from what physical scenario this Hilbert space emerges as there are numerous options. The Pauli matrices are, for example, associated to the spin angular momentum operator describing the spin, in the three directions $x,y,z$, of spin $\frac{1}{2}$ particles\cite{nielsen_chuang_2021}.

	 	\subsection{Qubits and Bloch representation}
	 	A qubit is simply a state in our Hilbert space thus
	 	\begin{equation}
	 		\ket{\psi} = a\ket{0} + b\ket{1}
	 	\end{equation}
	 	where $|a|^2+|b|^2 = 1$ due to normalisation condition.

	 	We can represent the above state using the Bloch representation\cite{nielsen_chuang_2021} by writing the state as
	 	\begin{equation}
	 		\ket{\psi} = \cos\left(\frac{\theta}{2}\right)\ket{0} + e^{i\phi}\sin\left(\frac{\theta}{2}\right)\ket{1}
	 	\end{equation}
	 	We can now associate a vector $\bm{r} = (1,\theta,\phi)\in\mathbb{R}^3$ ,in spherical polar coordinates, to this state to visualise it. The above \emph{Bloch vector} points on the unit ball.
	\end{comment}

 	\bibliography{refs}
 	\bibliographystyle{unsrt}
\end{document}