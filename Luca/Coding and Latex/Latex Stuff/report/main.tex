\documentclass{physics_article_B}

\mytitle{Approximating quantum many-body ground states with quantum circuits}
\myname{Luca Petru Ion, Nicholas Synesi}
\studentid{20177336, -}
\date{}


\begin{document}
\maketitle
\newpage

\begin{abstract}
(\emph{The abstract usually consists of a single paragraph of 100-200 words and is self-contained (i.e. no citations of references or mention of other parts of the article). The typical structure is one or two sentences each on the aim of the report, methods used, results found and discussion of the significance of the results}).
\end{abstract}

\tableofcontents
\newpage

\section{Introduction\label{intro}}
(\emph{The introduction will tell the reader what is included in the report, the background to the project, what other researchers have done in the area and how the work presented here relates to this wider body of research. The introduction will not repeat information from the abstract. It will then inform readers of the structure of the article (to provide a framework within which they can approach the rest of the report}). 

\section{Description of how the project was conducted\label{method}}
(\emph{In an experimental report this section is often referred to as the ‘methods section’. It should include a self-contained description of the methodology used) After the introduction, a project report will include a self-contained description of how the project was conducted and what methodology was followed. The reader should be assumed to have a similar physics background to the author, but be unfamiliar with the specific details of the project. Sufficient information should be given so that the reader could repeat the project if they wished. In a report it is necessary to be selective as to what is presented, do not include unnecessary detail (e.g. that equipment broke down and was replaced)}.

\section{Introduction to quantum computing and qubits}
\label{introduction_to_quantum_computing_and_qubits}
\subsubsection{Qbutis} % (fold)
\label{sub:qubits}

% subsubsection qbutis (end)
\subsection{Multi particle systems} % (fold)
\label{sub:qulti_particle}

% subsection multi_particle (end)
\subsection{Quantum gates} % (fold)
\label{sub:quantum_gates}

% subsection quantum_gates (end)
\subsection{Quantum circuits} % (fold)
\label{sub:quantum_circuits}

% subsection quantum_circuits (end)
\subsection{Tensors and diagrams} % (fold)
\label{sub:tensors_and_diagrams}

% subsection tensors_and_diagrams (end)
% section introduction_to_quantum_computing_and_qubits (end)

\section{Exact diagonalization} % (fold)
\label{sec:exact_diagonalization}
\subsection{Ising model} % (fold)
\label{sub:ising_model}

% subsection ising_model (end)
\subsection{Ground state} % (fold)
\label{sub:ground_state}

% subsection ground_state (end)
% section exact_diagonalization (end)

\section{Brick wall circuit} % (fold)
\label{sec:brick_wall_circuit}

% section section_name (end)

\section{Optimization process} % (fold)
\label{sec:optimization_process}

% section optimization_process (end)

\section{Entanglement} % (fold)
\label{sec:entanglement}
\subsection{Von Neumann entropy} % (fold)
\label{sub:von_neumann_entropy}

% subsection von_neumann_entropy (end)
\subsection{Mutual information} % (fold)
\label{sub:mutual_information}

% subsection mutual_information (end)
% section entanglement (end)

\section{Results\label{results}}
(\emph{The key findings of the investigation are reported here; the data that are presented have to be chosen carefully. Figures need to be drawn especially for the report. The analysis of uncertainties must be described. All numerical values found as a result of the investigation need to be given along with an estimate of the uncertainty, which is usually the standard deviation of the mean.})

\section{Discussion of Results\label{results}}
(\emph{Once the results have been presented it is important to discuss the scientific significance of the results, how the results compare to those published elsewhere and what are the strengths and limitations of the current investigation in relation to other work.})

\section{Conclusions\label{conclusions}}
(\emph{The purpose of this final section is to summarize the scientific study in a few paragraphs.})

\newpage
\noindent{\Large\bf Appendices}
\appendix
\section{\label{}}

\bibliography{BIB_FILE}
\bibliographystyle{unsrt}

\end{document}
